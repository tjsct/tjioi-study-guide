\chapter{Input/Output}

The following are some I/O examples in various languages to get you started. All programs read in a number $N$ from the first line, proceed to read in $N$ more numbers from the second line, and print their sum.

\section{Java}

\lstset{language=Java}

Here we use a \texttt{BufferedReader} instead of a \texttt{Scanner} because it's faster and more reliable in the contest environment.

\subsection{Standard I/O}

\begin{mylstlisting}
import java.util.*;
import java.io.*;

class sum {
    public static void main(String[] args) throws IOException {
        BufferedReader f = new BufferedReader(new InputStreamReader(System.in));
	    int N = Integer.parseInt(f.readLine()); // read whole line
	    int[] num = new int[N];
	    StringTokenizer st = new StringTokenizer(f.readLine()); // split line by white space
    	for(int k = 0; k < N; ++k) {
    		num[k] = Integer.parseInt(st.nextToken());
    	}
    	int sum = 0;
    	for(int k = 0; k < N; ++k) {
    		sum += num[k];
    	}
    	System.out.println(sum);
    	System.exit(0);
    }
}
\end{mylstlisting}

\subsection{File I/O}

\begin{mylstlisting}
import java.util.*;
import java.io.*;

class sum {
    public static void main(String[] args) throws IOException {
        BufferedReader f = new BufferedReader(new FileReader("sum.in"));
		PrintWriter out = new PrintWriter(new BufferedWriter(new FileWriter("sum.out")));
	    int N = Integer.parseInt(f.readLine()); // read whole line
	    int[] num = new int[N];
	    StringTokenizer st = new StringTokenizer(f.readLine()); // split line by white space
    	for(int k = 0; k < N; ++k) {
    		num[k] = Integer.parseInt(st.nextToken());
    	}
    	int sum = 0;
    	for(int k = 0; k < N; ++k) {
    		sum += num[k];
    	}
    	out.println(sum);
    	out.close(); // don't forget this!
    	System.exit(0);
    }
}
\end{mylstlisting}

\section{C++}

\lstset{language=C++}

Here we use C++-style I/O. You may alternatively use C-style I/O. 

\subsection{Standard I/O}

The first two lines are to speed up input. They are considered bad coding practice outside of the contest environment but are essential to get your times down if I/O is large. Note, however, that if you unlink with C-style I/O, you may not use \texttt{scanf()} and \texttt{printf()}, etc.

\begin{mylstlisting}
#include <iostream>
#include <fstream>

int num[100005];

int main() {
    std::ios_base::sync_with_stdio(0); // unlink C-style I/O
    std::cin.tie(0); // unlink std::cout
	std::cin >> N;
	for(int k = 0; k < N; ++k) {
		std::cin >> num[k];
	}
	int sum = 0;
	for(int k = 0; k < N; ++k) {
		sum += num[k];
	}
	cout << sum << "\n";
	return 0;
}
\end{mylstlisting}

\subsection{File I/O}

\begin{mylstlisting}
#include <iostream>
#include <fstream>

int num[100005];

int main() {
	std::ifstream fin("palpath.in");
	std::ofstream fout("palpath.out");
	fin >> N;
	for(int k = 0; k < N; ++k) {
		fin >> num[k];
	}
	int sum = 0;
	for(int k = 0; k < N; ++k) {
		sum += num[k];
	}
	fout << sum << "\n";
	fin.close();
	fout.close(); // don't forget this!
	return 0;
}
\end{mylstlisting}

\section{Python}

\lstset{language=Python}

\subsection{Standard I/O}

\begin{mylstlisting}
n = int( input() ) # input() grabs the whole line
nums = input().strip() # removes extra spaces at beginning and end, also \n
nums = nums.split() # splits at the spaces to turn it into an array of strings
nums = [int(stng) for stng in nums] #turn them into ints
print( sum( nums ) )
\end{mylstlisting}

\subsection{File I/O}

\begin{mylstlisting}
file = open('input.txt', 'r') # r for read
out = open('output.txt', 'w') # w for write

n = int(file.readline().strip()) # strip() isn't always necessary, but it's a good habit
nums = file.readline().strip()
nums = nums.split() # splits at the spaces to turn it into an array of strings
nums = [int(stng) for stng in nums] #turn them into ints
out.write( str( sum( nums ) ) + '\n' ) # str() and + are necessary because write() takes only a single string

file.close()
out.close()
\end{mylstlisting}